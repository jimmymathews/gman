
%\usepackage{tikz-cd, amscd}


%\usepackage{amscd}

%\usepackage{hyperref} % this doesnt work, dont know why

\usepackage{amsmath} 

\theoremstyle{plain}% default
\newtheorem{theorem}{Theorem}[section]
\newtheorem{lemma}[theorem]{Lemma}
\newtheorem{proposition}[theorem]{Proposition}%[section]
\newtheorem{cor}[theorem]{Corollary}%[section]

\theoremstyle{definition}
\newtheorem{definition}[theorem]{Definition}%[section]
\newtheorem*{example}{Example}

\theoremstyle{remark}
\newtheorem*{rmrk}{Remark}



\usepackage[all]{xy}


\usepackage{enumitem}

\usepackage{float}
\usepackage{etoolbox}
\patchcmd{\subsection}{\bfseries}{\bf}{}{}
\patchcmd{\subsection}{-.5em}{.5em}{}{}

\patchcmd{\subsubsection}{\bfseries}{\bf}{}{}
%\patchcmd{\subsubsection}{-.5em}{.1em}{}{}


%\usepackage[spanish,activeacute]{babel} %\usepackage[utf8]{inputenc}
\usepackage{graphicx} %\renewcommand{\familydefault}{ppl}
%\textwidth=465pt\textheight=600pt\oddsidemargin=0pt\evensidemargin=0pt\topmargin=10pt\headheight=0pt\headsep=30pt
\newcommand{\iso}{\stackrel{\sim}{\to}}
\newcommand{\cR}{\mathcal R}
\newcommand{\cA}{\mathcal A}
\newcommand{\cB}{\mathcal B}
\newcommand{\cC}{\mathcal C}
%
\newcommand{\cW}{\mathcal W}
\newcommand{\BOM}{\mathbf{\Omega}}
\newcommand{\BLA}{\mathbf{\Lambda}}

\newcommand{\tH}{\widetilde{H}}

\newcommand{\Span}{\mathrm{Span}}

\newcommand{\ed}{x}
\newcommand{\II}{\mathrm I}

\newcommand{\lc}{locally convex}


\newcommand{\eqns}{(\ref{eqns})}

\newcommand{\eqnsss}{(\ref{eqnsss})}

\newcommand{\Q}{Q^5}
\newcommand{\M}{M^4}
\newcommand{\N}{N^5}
\newcommand{\zm}{\ast}

\newcommand{\eto}{\xrightarrow{\sim}}

%{\overset{.|\sim}{\to}}


\newcommand{\tG}{\tilde \Gamma}
\newcommand{\PC}{\mathcal{PC}}


%\usepackage{amsthm} 
%\newtheorem{theorem}{Theorem}
%\newtheorem{lemma}{Lemma}
%\newtheorem{proposition}{Proposition}
%\newtheorem{definition}{Definition} 
%\newtheorem{cor}{Corollary} 
\renewcommand{\>}{\rangle}
\newcommand{\<}{\langle} 
\newcommand{\w}{{\mathbf w}}
\newcommand{\bv}{{\mathbf v}} 
\newcommand{\m}{\zeta}
\newcommand{\x}{{\mathbf x}}
\newcommand{\y}{{\mathbf y}}
\newcommand{\tO}{\widetilde{\mathbb O}}
\newcommand{\Om}{{\Omega}}
\newcommand{\X}{\times}
\newcommand{\G}{{\rm G}_2}
\newcommand{\Aut}{{\rm Aut}}
\newcommand{\Der}{{\rm Der}}
%\newcommand{\rk}{\n{\em Remark. }}
%\newcommand{\rks}{{\em Remarks. }}
\newcommand{\fip}{\phi^\prime} 
\newcommand{\tr}{\mbox{tr}}
\newcommand{\alt}{\mbox{alt}} \newcommand{\interior}{\mbox{int}}
\renewcommand{\u}{{\bf u}} 
\newcommand{\g}{{\mathfrak{g}}}
\newcommand{\gl}{{\mathfrak{gl}}}
\renewcommand{\sp}{{\mathfrak{sp}}} 
\renewcommand{\a}{{\bf a}}
\renewcommand{\b}{{\bf b}}
%\renewcommand{\c}{{\bf c}}
\newcommand{\bfc}{{\bf c}}



\renewcommand{\d}{{\bf d}}
\renewcommand{\o}{ \boldsymbol{\omega}}
\renewcommand{\O}{ \boldsymbol{\Omega}}
%\renewcommand{\i}{{\iota}} \renewcommand{\L}{{\Lambda}}
\renewcommand{\P}{\mathbb{P}} 

\newcommand{\PT}{\P^2}
\newcommand{\pf}{{\n\em Proof. }}
\newcommand{\eq}{(\ref{main_eqns1}) }
\newcommand{\eqq }{(\ref{main_eqns2}) }

\newcommand{\tQ}{\overline{Q}^5}
\newcommand{\tC}{\widetilde{C}}
\renewcommand{\Im}{{\rm Im}}
\renewcommand{\Re}{{\rm Re}}
\newcommand{\uh}{ad(\u)}
\newcommand{\vh}{ad(\v)}
 \newcommand{\wnab}{\widetilde\nabla}
 \newcommand{\rh}{\rho_h}
 

 \newcommand{\Ah}{\widehat{A}}
\newcommand{\ah}{\widehat{a}}
\newcommand{\Bh}{\widehat{B}}
\newcommand{\bh}{\widehat{b}}

 \newcommand{\CC}{\mathcal{C}}
 
\newcommand{\RP}{\R\P}
\newcommand{\RPt}{\RP^2}
\newcommand{\RPts}{\RP^{2*}}
\newcommand{\Rts}{(\R^3)^*}
\newcommand{\Rt}{\R^3}


\newcommand{\ent}{\Longrightarrow}
\newcommand{\st}{\, | \,}

\newcommand{\PTS}{{\PT}^*} 
\newcommand{\R}{\mathbb{R}}
\newcommand{\Rtt}{\R^{3,3}} 

\newcommand{\C}{\mathbb{C}} 
\newcommand{\Z}{\mathbb{Z}}
\newcommand{\T}{\mathbb{T}}
\renewcommand{\H}{\mathbb{H}}


\newcommand{\E}{{\mathbb E}} 
\renewcommand{\S}{{\mathbb S}}
\newcommand{\K}{{\mathbb K}} 
\renewcommand{\qed}{\hfill\mbox{$\Box$}}

\newcommand{\D}{{\mathcal D}} 
\newcommand{\tD}{\overline{\mathcal D}} 
\newcommand{\SL}{\mathrm{SL}}

\newcommand{\GL}{\mathrm{GL}}
\newcommand{\GLt}{\GL_2(\R)}
\newcommand{\SLt}{{\SL_2(\R)}} 
\newcommand{\SLth}{{\SL_3(\R)}} 
\newcommand{\SO}{\mathrm{SO}}

\newcommand{\slt}{\mathfrak{sl}_2(\R)} 
\newcommand{\slth}{\mathfrak{sl}_3(\R)} 
\renewcommand{\sl}{\mathfrak{sl}} 

\newcommand{\glt}{\mathfrak{gl}_2(\R)} 
\newcommand{\glf}{\mathfrak{gl}_4(\R)}
\newcommand{\h}{{\mathfrak h}} 
\newcommand{\I}{{\mathbb I^3}}
\newcommand{\prop}{\mn{\bf Proposition. }}


\newcommand{\spk}{\mathfrak{sp}_k} \newcommand{\spo}{\mathfrak{sp}_1} 
\newcommand{\so}{\mathfrak{so}} \newcommand{\tensor}{\otimes}
\newcommand{\End}{{\rm End}} 
\newcommand{\Ker}{{\rm Ker}}
\newcommand{\Ad}{{\rm Ad}} 
\renewcommand{\mod}{\hbox{mod }}
\newcommand{\marco}[1]{\framebox{$\displaystyle #1 $}}

\newcommand{\n}{\noindent} 
\newcommand{\bs}{\bigskip}
\newcommand{\ms}{\medskip} 
\newcommand{\sn}{\smallskip\n} 

\newcommand{\mn}{\medskip\noindent}
\newcommand{\bn}{\bs\n} 
\newcommand{\bnbf}{\bs\n\bf}

\newcommand{\ITEM}{\mn\item } 
\newcommand{\p}{{\mathbf p} }
\newcommand{\ph}{{\widehat \p} } 
\newcommand{\tj}{{\tilde j} } 
\newcommand{\q}{\mathbf q } 
\newcommand{\qh}{{\widehat \q} } 
\newcommand{\qb}{{\overline \q} } 
%%%%%%%%%%%%%%%%%%
\usepackage{etoolbox}
\makeatletter
\let\old@tocline\@tocline
\let\section@tocline\@tocline
% Insert a dotted ToC-line for \subsection and \subsubsection only
\newcommand{\subsection@dotsep}{4.5}
\newcommand{\subsubsection@dotsep}{4.5}
\patchcmd{\@tocline}
  {\hfil}
  {\nobreak
     \leaders\hbox{$\m@th
        \mkern \subsection@dotsep mu\hbox{.}\mkern \subsection@dotsep mu$}\hfill
     \nobreak}{}{}
\let\subsection@tocline\@tocline
\let\@tocline\old@tocline

\patchcmd{\@tocline}
  {\hfil}
  {\nobreak
     \leaders\hbox{$\m@th
        \mkern \subsubsection@dotsep mu\hbox{.}\mkern \subsubsection@dotsep mu$}\hfill
     \nobreak}{}{}
\let\subsubsection@tocline\@tocline
\let\@tocline\old@tocline

\let\old@l@subsection\l@subsection
\let\old@l@subsubsection\l@subsubsection

\def\@tocwriteb#1#2#3{%
  \begingroup
    \@xp\def\csname #2@tocline\endcsname##1##2##3##4##5##6{%
      \ifnum##1>\c@tocdepth
      \else \sbox\z@{##5\let\indentlabel\@tochangmeasure##6}\fi}%
    \csname l@#2\endcsname{#1{\csname#2name\endcsname}{\@secnumber}{}}%
  \endgroup
  \addcontentsline{toc}{#2}%
    {\protect#1{\csname#2name\endcsname}{\@secnumber}{#3}}}%

% Handle section-specific indentation and number width of ToC-related entries
\newlength{\@tocsectionindent}
\newlength{\@tocsubsectionindent}
\newlength{\@tocsubsubsectionindent}
\newlength{\@tocsectionnumwidth}
\newlength{\@tocsubsectionnumwidth}
\newlength{\@tocsubsubsectionnumwidth}
\newcommand{\settocsectionnumwidth}[1]{\setlength{\@tocsectionnumwidth}{#1}}
\newcommand{\settocsubsectionnumwidth}[1]{\setlength{\@tocsubsectionnumwidth}{#1}}
\newcommand{\settocsubsubsectionnumwidth}[1]{\setlength{\@tocsubsubsectionnumwidth}{#1}}
\newcommand{\settocsectionindent}[1]{\setlength{\@tocsectionindent}{#1}}
\newcommand{\settocsubsectionindent}[1]{\setlength{\@tocsubsectionindent}{#1}}
\newcommand{\settocsubsubsectionindent}[1]{\setlength{\@tocsubsubsectionindent}{#1}}

% Handle section-specific formatting and vertical skip of ToC-related entries
% \@tocline{<level>}{<vspace>}{<indent>}{<numberwidth>}{<extra>}{<text>}{<pagenum>}
\renewcommand{\l@section}{\section@tocline{1}{\@tocsectionvskip}{\@tocsectionindent}{}{\@tocsectionformat}}%
\renewcommand{\l@subsection}{\subsection@tocline{1}{\@tocsubsectionvskip}{\@tocsubsectionindent}{}{\@tocsubsectionformat}}%
\renewcommand{\l@subsubsection}{\subsubsection@tocline{1}{\@tocsubsubsectionvskip}{\@tocsubsubsectionindent}{}{\@tocsubsubsectionformat}}%
\newcommand{\@tocsectionformat}{}
\newcommand{\@tocsubsectionformat}{}
\newcommand{\@tocsubsubsectionformat}{}
\expandafter\def\csname toc@1format\endcsname{\@tocsectionformat}
\expandafter\def\csname toc@2format\endcsname{\@tocsubsectionformat}
\expandafter\def\csname toc@3format\endcsname{\@tocsubsubsectionformat}
\newcommand{\settocsectionformat}[1]{\renewcommand{\@tocsectionformat}{#1}}
\newcommand{\settocsubsectionformat}[1]{\renewcommand{\@tocsubsectionformat}{#1}}
\newcommand{\settocsubsubsectionformat}[1]{\renewcommand{\@tocsubsubsectionformat}{#1}}
\newlength{\@tocsectionvskip}
\newcommand{\settocsectionvskip}[1]{\setlength{\@tocsectionvskip}{#1}}
\newlength{\@tocsubsectionvskip}
\newcommand{\settocsubsectionvskip}[1]{\setlength{\@tocsubsectionvskip}{#1}}
\newlength{\@tocsubsubsectionvskip}
\newcommand{\settocsubsubsectionvskip}[1]{\setlength{\@tocsubsubsectionvskip}{#1}}

% Adjust section-specific ToC-related macros to have a fixed-width numbering framework
\patchcmd{\tocsection}{\indentlabel}{\makebox[\@tocsectionnumwidth][l]}{}{}
\patchcmd{\tocsubsection}{\indentlabel}{\makebox[\@tocsubsectionnumwidth][l]}{}{}
\patchcmd{\tocsubsubsection}{\indentlabel}{\makebox[\@tocsubsubsectionnumwidth][l]}{}{}

% Allow for section-specific page numbering format of ToC-related entries
\newcommand{\@sectypepnumformat}{}
\renewcommand{\contentsline}[1]{%
  \expandafter\let\expandafter\@sectypepnumformat\csname @toc#1pnumformat\endcsname%
  \csname l@#1\endcsname}
\newcommand{\@tocsectionpnumformat}{}
\newcommand{\@tocsubsectionpnumformat}{}
\newcommand{\@tocsubsubsectionpnumformat}{}
\newcommand{\setsectionpnumformat}[1]{\renewcommand{\@tocsectionpnumformat}{#1}}
\newcommand{\setsubsectionpnumformat}[1]{\renewcommand{\@tocsubsectionpnumformat}{#1}}
\newcommand{\setsubsubsectionpnumformat}[1]{\renewcommand{\@tocsubsubsectionpnumformat}{#1}}
\renewcommand{\@tocpagenum}[1]{%
  \hfill {\mdseries\@sectypepnumformat #1}}

% Small correction to Appendix, since it's still a \section which should be handled differently
\let\oldappendix\appendix
\renewcommand{\appendix}{%
  \leavevmode\oldappendix%
  \addtocontents{toc}{%
    \protect\settowidth{\protect\@tocsectionnumwidth}{\protect\@tocsectionformat\sectionname\space}%
    \protect\addtolength{\protect\@tocsectionnumwidth}{2em}}%
}
\makeatother

% #1 (default is as required)

% #2

% #3
\makeatletter
\settocsectionnumwidth{2em}
\settocsubsectionnumwidth{2.5em}
\settocsubsubsectionnumwidth{3em}
\settocsectionindent{1pc}%
\settocsubsectionindent{\dimexpr\@tocsectionindent+\@tocsectionnumwidth}%
\settocsubsubsectionindent{\dimexpr\@tocsubsectionindent+\@tocsubsectionnumwidth}%
\makeatother

% #4 & #5
\settocsectionvskip{10pt}
\settocsubsectionvskip{0pt}
\settocsubsubsectionvskip{0pt}

% #6 & #7
% See #3

% #8
\renewcommand{\contentsnamefont}{\bfseries\Large}

% #9
\settocsectionformat{\bfseries}
\settocsubsectionformat{\mdseries}
\settocsubsubsectionformat{\mdseries}
\setsectionpnumformat{\bfseries}
\setsubsectionpnumformat{\mdseries}
\setsubsubsectionpnumformat{\mdseries}

% #10
% Insert the following command inside your text where you want the ToC to have a page break
\newcommand{\tocpagebreak}{\leavevmode\addtocontents{toc}{\protect\clearpage}}

% #11
\let\oldtableofcontents\tableofcontents
\renewcommand{\tableofcontents}{%
  \vspace*{-\linespacing}% Default gap to top of CONTENTS is \linespacing.
  \oldtableofcontents}

\setcounter{tocdepth}{3}

%%%%%%%%%%%%%%%%%%%%



\newcommand{\mm}{\mathfrak m}
\newcommand{\wH}{\widehat H}
\newcommand{\wG}{\widehat G}
\newcommand{\wtheta}{\widehat\theta}
\newcommand{\wPhi}{\theta}
\newcommand{\wh}{\widehat\h}
\newcommand{\weta}{\hat\eta}

\newcommand{\om}[2]{\omega^{#1}_{\;#2}}
\newcommand{\Ph}[2]{\Phi^{#1}_{\;#2}}

\newcommand{\tth}[2]{\theta^{#1}_{\;#2}}
\newcommand{\vph}[2]{\varphi^{#1}_{\;#2}}


\newcommand{\V}{\mathcal V}
\newcommand{\etau}{\eta^\bullet}
\newcommand{\etad}{\eta_\bullet}
\newcommand{\be}{\begin{equation}}
\newcommand{\ee}{\end{equation}}
\newcommand{\hh}{\widehat h}
\newcommand{\mrI}{\mathrm I}
\newcommand{\stars}{\ms\centerline{$* \qquad * \qquad *$ }}
\newcommand{\met}{ \mathbf{ g}}

\newcommand{\da}{\downarrow}
\newcommand{\lda}{\Bigg\downarrow}
\newcommand{\lra}{\longrightarrow}
\newcommand{\wB}{\widehat B}

\newcommand{\wF}{\widehat F}


