\section{Pseudo-riemannian geometry in signature $(2,2)$}\label{pr}

 In this section we relate the geometry of the (2,3,5)-distribution $(\Q, \D)$ given by Eqns.~(\ref{eqns}) to 4-dimensional  conformal  geometry, by giving $\Q$ the structure of a principal $\R^*$-bundle $\Q\to \M$, the  ``pseudo-Hopf-fibration'', inducing on  $\M$  a  split-signature pseudo-riemannian metric $\met$, which we call the ``dancing metric''; the  name is due to an amusing alternative  definition  of the {\em conformal class} $[\met]$ (see Def.~\ref{def_danc} of Sect.~\ref{sec_danc}).


We then show in Thm.~\ref{eleven} (Sect.~\ref{tang_SD}), using the Maurer-Cartan structure equations of $\SLth$, that the projection $\Q\to \M$ establishes a bijection between  integral curves  in  $(\Q,\D)$   and  (non-degenerate) null-curves in $(\M,[\met])$ with  {\em parallel  self-dual tangent null 2-plane.} 

A more conceptual explanation to Thm.~ \ref{eleven} is given in Thm.~\ref{ident}, where we  show  that  $(\Q, \D)$ can  be naturally embedded in  the  total space of the {\em self-dual twistor fibration} $\RP^1\to \T^+(\M)\to \M$ associated with $(\M,[\met])$, equipped with its canonical {\em twistor distribution} $\D^+$, as introduced in \cite{AN}. The non-integrability of $\D$ is then seen to be due  to the non-vanishing of the self-dual Weyl tensor of $\met$.

\subsection{The pseudo-Hopf-fibration and the dancing metric} 

\subsubsection{First definition of the dancing metric}\label{first}
Recall from Sect.~\ref{TwoOne} that $\Q=\{(\q,\p)|\p\q=1\}\subset \Rtt$ (the ``unit pseudo-sphere"). To each pair $(\q,\p)\in\Q$ we assign the pair $\Pi(\q, \p)=([\q], [\p])=(q, p)\in \RPt\times\RPts$, where  $q\in \RPt$, $p\in\RPts$ are the points with homogeneous coordinates $\q, \p$ (resp.). Let $\I\subset\RPt\times\RPts$ be the subset of pairs $(q,p)$ given in homogeneous coordinates by the equation $\p\q=0$, also called {\em incident pairs}  (the name comes from the geometric interpretation of such a pair  as  a ({\em point, line}) pair, such that the {\em line} passes through  the {\em point}; more on this in  Sect. 5).  It is easy to see from the equation $\p\q=0$ that $\I$ is a 3-dimensional  closed submanifold of $\RPt\times\RPts$.  Its   complement  $$\M:=(\RPt\times\RPts)\setminus \I$$ is  the set of {\em non-incident point-line pairs},   a connected  open dense  subset of $\RPt\times\RPts$. Clearly, if $\p\q=1$ then $([\q], [\p])\not\in\I$, thus $\Pi:\Q\to \M$ is well defined. 


 Define  an $\R^*$-action on $\Q$, where $\lambda\in \R^*$  acts   by 
 \be\label{Raction}
 (\q, \p)\mapsto
(\lambda\q, \p/\lambda), \quad \lambda\in\R^*.\ee 
This is a free $\R^*$-action whose  orbits are precisely the fibers of 
$$\Pi:\Q\to \M, \quad (\q, \p)\mapsto([\q], [\p]).$$
That is, $\Pi$ is a principal $\R^*$-fibration.  Now the quadratic form $\p\q$ defines on  $\Rtt$    a flat split-signature metric, whose restriction to $\Q\subset \Rtt$ is a  $(2,3)$-signature metric. Furthermore,   the principal $\R^*$-action on $\Q$ is by isometries, generated by a  negative definite vector field. 
 Combining these we get:



\begin{proposition}[\bf definition of the dancing metric]\label{dancing}
Restrict the flat split-signature metric $-2d\p \, d\q$ on $\Rtt$ to $\Q$. Then there is a unique pseudo-riemannian metric $\met$ on $\M$, of signature $(2,2)$, rendering $\Pi:\Q\to \M$  a pseudo-riemannian  submersion. We call $\met$  the {\em dancing metric}. 
 \end{proposition}
 
\begin{rmrk} The factor $-2$ in the above definition is not essential and is introduced merely for simplifying later explicit formulas for $\met$. 
\end{rmrk}


\begin{rmrk} This definition  is analogous to the definition  of the Fubini-Study metric  on $\C\P^2$ via the (usual)  Hopf fibration $S^1\to S^5\to\C\P^2$. In fact, $\M$ is referred to  by some authors as  the ``para-complex projective plane'' and  $\met$ as the ``para-Fubini-Study metric"  \cite{A,CFG}. 
\end{rmrk}

Using the $\SLth$-invariance of $\met$ it is not difficult to come up with an explicit formula for $\met$ in homogenous coordinates  $\q, \p$ on $\RPt, \RPts$ (resp.).
\begin{proposition}\label{homo}
Let $\widetilde \Pi:\Rtt\setminus \{\p\q=0\}\to \M, $ $(\q, \p)\mapsto ([\q],[\p])$. Then 
\be\label{homog}
\widetilde \Pi^*\met=-2{(\q\times d\q)(\p\times d\p)\over (\p\q)^2}.
\ee
\end{proposition}

\begin{proof} The expression on the right of Eqn.~(\ref{homog}) is a quadratic 2-form, $\R^*\times \R^*$-invariant, $\widetilde\Pi$-horizontal (vanishes on
 $\widetilde\Pi$-vertical vectors) and $\SLth$-invariant. It thus descends to an 
 $\SLth$-invariant quadratic 2-form on $M$. 
By examining the isotropy representation of the stabilizer of a point in 
 $M$ (Eqn.~(\ref{isotropy}) below) we see that $M$ admits a unique $\SLth$ quadratic 2-form, 
 up to a constant multiple. It is thus sufficient to verify  the formula  on a single non-null vector, 
 say $e_1-e^1\in T_{(e_3, e^3)}Q.$ We omit this (easy) verification. 
 \end{proof}
 
\begin{rmrk} Using standard vector identities, formula~(\ref{homog}) can be rewritten also as 
 \be\label{homog_bis}
\widetilde \Pi^*\met=-2{(\p\q)(d\p\,  d \q)-(\p\, d\q) (d\p \,\q)\over (\p\q)^2}.
\ee
\newcommand{\z}{\mathbf z}
An advantage of this formula is that it makes sense in higher dimensions, definining the ``para-Fubini-Study" metric on $\left[\R^{n+1,n+1}\setminus\{\p\q=0\}\right]/(\R^*\times \R^*)$. It also compares nicely  with the usual formula for the (standard) Fubini-Study metric $\met_{FS}$ on $\C P^n=\left[\C^{n+1}\setminus\{0\}\right]/\C^*$, given  in homogenous coordinates $\z=(z_0, \ldots z_n)^t\in\C^{n+1}$, $\z^*:=\bar\z^t$, by
$$\widetilde\Pi^*\met_{FS}={(\z^*\z) (d\z^* \,d\z)-( \z^*\, d\z)( d\z^*\, \z)\over (\z^*\z)^2}.$$
\end{rmrk} 



 We give later  three  more explicit formulas for  $\met$: in Prop.~\ref{all}a $\met$ is expressed  in terms of the Maurer-Cartan form  of $\SLth$, analogous to a formula for the Fubini-Study metric on $\C P^n$ in terms of the Maurer-Cartan form of $\mathrm{SU}_{n+1}$. In Prop.~\ref{cr} we give a ``cross-ratio"  formula for $
\met$. In Sect.~\ref{simple} we derive a simple formula  in local coordinates for the {\em conformal class}  $[\met]$, using the ``dancing condition''.   

\subsubsection{Orientation}\label{orientation} We define an orientation on $\M$ via its {\em para-complex structure}. Namely, using the decomposition $T_{(q,p)}\M=T_q\RPt\oplus T_p\RPts$, define  $K: TM\to TM$ by $K(q',p')= (q',-p')$.  A {\em para-complex basis} for $T_{(q,p)}\M$ is then an ordered   basis of the form $(v_1, v_2, K v_1,  K v_2)$. One can check easily that any two such bases are related by a matrix with positive determinant,  hence these bases give a well-defined orientation on $\M$. See Prop.~\ref{all}c below for an alternative definition via a volume form on $\M$, written in terms of the components of the Maurer-Cartan form of $\SLth$.    

\subsubsection{Some properties of the dancing metric}\label{proper}

The dancing metric has remarkable properties. We group in the next theorem some of them.  

\begin{theorem} \label{Ten}

\sn\begin{enumerate}[leftmargin=*]\setlength\itemsep{5pt}
\item  $(\M,\met)$ is the  homogeneous  symmetric space  $\SLth/H$, where   $H\simeq \GL_2(\R)$ (the precise subgroup $H$ is described below in Sect.~\ref{proofs}).  The $\SLth$-action on $\M$ is induced from the standard action on $\Rtt$, $([\q], [\p])\mapsto ([g\q], [\p g^{-1}]).$ The $\GL_2(\R)$-structure endows $\M$ with a structure of a {\em para-Kahler} manifold. 

\item  $(\M,\met)$  is a  complete, Einstein,   irreducible, pseudo-riemannian 4-manifold of signature $(2,2)$. 
It   is self-dual (with respect to the above orientation), i.e. its anti-self-dual Weyl tensor $\mathcal W^-\equiv0$,  but  is not conformally flat; its self-dual Weyl curvature tensor $\mathcal W^+$ is nowhere vanishing, of Petrov type $D$.  



\item The splitting $T_{(q,p)}\M=T_q\RPt\oplus T_p\RPts$ equips $\M$ with a pair of complementary  null, self-dual,  parallel,  integrable,   rank 2 distributions. Their integral leaves generate a pair of foliations of $\M$ by  totally geodesic self-dual null surfaces, the fibers of the double  fibration
%
$$    \xymatrix{
        &\M  \ar[dl]_\pi \ar[dr]^{\bar\pi} & \\
                   \RPt&            & \RPts }
$$

\item $\M$ admits a 3-parameter family of anti-self-dual totally geodesic
 null surfaces, naturally parametrized by the incidence variety  $\I:=\{(\bar q, \bar p)|\bar q\in \bar p\}\subset\RPt\times\RPts.$
For  each incident pair $(\bar q, \bar p)\in\I$, the corresponding surface is the set $\Sigma_{\bar q, \bar p}$ of non-incident pairs $(q,p)$ such that $q\in\bar p$ and $\bar q\in p$. 
\end{enumerate}
\end{theorem}
\begin{figure}[h]\centering
\includegraphics[width=0.3\textwidth]{N}
\caption{The definition of   $\Sigma_{\bar q, \bar p}$}
\end{figure}
%
\begin{rmrk} The last  point (4) can be reformulated as follows: let $\N\subset  \M\times\I$ be defined via the incidence diagram above, i.e.
$$\N=\{(q,p,\bar q, \bar p)| q\not\in p, \bar q\in\bar  p, q\in\bar p, \bar q\in  p\}\subset\RPt\times \RPts\times\RPt\times \RPts.$$ 
%
Then $\N$ is a 5-dimensional submanifold of $\M\times \I $, equipped with the  double fibration
%
\be\label{double}    
\xymatrix{
&\N  \ar[dl]_{\pi_{12}} \ar[dr]^{\pi_{34}} & \\
\M& & \I }
\ee
%
The right-hand fibration $\pi_{34}:\N\to\I$ foliates $\N$ by 2-dimensional surfaces, each of which projects via $\pi_{12}:\N\to \M$
to one of the surfaces  $\Sigma_{\bar q, \bar p}$. That is, $\Sigma_{\bar q, \bar p}=\pi_{12}\left(\pi_{34}^{-1}(\bar q, \bar p)\right).$ The left-hand fibration $\pi_{12}:\N\to\M$ foliates $\N$  by projective lines and  can be naturally identified with the {\em anti-self-dual twistor fibration} $\T^-\M\to\M$ associated with $(\M,[\met])$ (see Sect.~\ref{twist}). The fibers of $\pi_{34}$ then correspond to the integral  leaves of the anti-self-dual twistor distribution $\D^-$, which is integrable in our case, due to the vanishing of $\mathcal W^-$ (see Cor.~\ref{grancor}, Sect.~\ref{proofs}  below).  
\end{rmrk}

 Most claims of this theorem can be found in  various sources in the literature (see e.g.~\cite{A} and the many references within). Using  the Maurer-Cartan equations of $\SLth$ (Sect.~\ref{proofs}), it is quite straightforward to prove these results. Alternatively, one can write down explicitly the dancing metric  in local coordinates (Sect.~\ref{simple}) and let a computer calculate curvature, symmetries etc. 

\subsection{Rudiments of 4-dimensional geometry in split-signature}
\subsubsection{Linear algebra}\label{linalg}  

Let  $V$ be  an oriented 4-dimensional real vector space equipped  with  a quadratic form $\<\,,\>$
 of signature $(++-\,-)$. It is convenient to introduce  {\em null bases} in such a $V$. This is a basis $\{e_1, e_2, e^1, e^2\}\subset V$  
such that 
$$\<e_a, e_b\>=\<e^a, e^b\>=0, \;  \<e^a, e_b\>=\delta^a_b, \qquad a,b=1,2.$$ 
%
Note that if $\{\ed^1, \ed^2,  \ed_1,  \ed_2 \}\subset V^*$ is the {\em dual basis} to a null basis, i.e. $\ed^a(e^b)=\ed_a(e_b)=0,  
\ed^a(e_b)=\ed_b(e^a)=\delta^a_b,$ then 
\be\label{metric}\<\,,\>=2(\ed^1 \ed_1+\ed^2 \ed_2).
\ee

\begin{rmrk} Our convention is that the symmetric tensor product $x y\in S^2\,V^*$ of two elements  $x,y\in V^*$   is the symmetric bilinear form 
%
\be (x y)(v,w):=[x(v)y(w)+y(v)x(w)]/2, \quad v, w\in V. 
\ee
\end{rmrk}

Now let  $vol:=\ed^1\wedge\ed^2\wedge\ed_1\wedge\ed_2\in \Lambda^4\,V^*$  and  $*:\Lambda^2\,V^*\to \Lambda^2\,V^*$ the  corresponding Hodge dual, satisfying  $\alpha\wedge*\beta=\<\alpha,\beta\>vol$, $\alpha, \beta\in \Lambda^2(V^*)$.  Then  $*^2=1$ and one has the splitting 
\be\label{hodge}\Lambda^2\,V^*=\Lambda^2_+\,V^*\oplus \Lambda^2_-\,V^*,\ee
where $\Lambda^2_\pm\,V^*$  are the $\pm 1$ eigenspaces of $*$, called   SD (self-dual) and the  ASD (anti-self-dual) 2-forms (resp.). 

  Let $\SO_{2,2}\subset \GL(V)$ be the corresponding orientation-preserving orthogonal 
group and $\so_{2,2}\subset \End\,V$ its Lie algebra. With respect to a null basis, the matrices of elements in $\so_{2,2}$  are of the form 
\be\label{sott}
\left(\begin{array}{cc}
A&B\\
C&-A^t
\end{array}\right), \quad A,B,C\in Mat_{2\times 2}(\R), \; B^t=-B, \; C^t=-C.
\ee
%






There is a natural isomorphism (equivalence of $\SO_{2,2}$-representations)
    \be\label{iso2}\so_{2,2}\iso\Lambda^2\,V^*,\quad T\mapsto {1\over 2}\<\,\cdot\, ,T\,\cdot\,\>
\ee
and  a Lie algebra decomposition $\so_{2,2}=\sl^+_2(\R)\oplus\sl^-_2(\R),$ given by
$$
{\footnotesize
 \left(\begin{array}{cc}
A&B\\
C&-A^t
\end{array}\right)=\left(\begin{array}{cc}
A_0&0\\
0&-A_0^t
\end{array}\right)+\left(\begin{array}{cc}
{\tr A\over 2}\II&B\\
C&-{\tr A\over 2}\II
\end{array}\right), \quad A_0=A-{\tr A\over 2}\II\in\sl_2(\R), 
}$$ 
matching  the decomposition of Eqn.~(\ref{hodge}), i.e. $\sl_2^\pm(\R)\iso \Lambda^2_\pm\,V^*.$


 Given a 2-plane $W\subset V$ pick a basis $\theta^1, \theta^2$  of the annihilator $W^0\subset V^*$ and  let $\beta=\theta^1\wedge\theta^2$. If we pick another basis of $W^0$ then $\beta $ is multiplied by a non-zero constant (the determinant of the matrix of change of basis), hence $\R\beta\subset \Lambda^2(V^*)$ is well-defined in terms of $W$ alone. This defines the {\em Pl\"ucker embedding}  of the grassmanian of 2-planes $Gr(2,V)\hookrightarrow\P(\Lambda^2\,V^*)\simeq \RP^5$. Its  image is given in homogeneous coordinates by the quadratic equation $\beta\wedge\beta=0$. We say that a 2-plane $W$ is SD (self-dual) if $\R\beta\subset \Lambda^2_+\,V^*$, and ASD (anti-self-dual)  if $\R\beta\subset \Lambda^2_-\,V^*$. We denote by 
 $$\T^+V:=\{W\subset V\st W\hbox{ is a SD 2-plane}\}.$$
Using the Pl\"ucker embedding,  $\T^+V$   is naturally  identified  with the conic in $\P(\Lambda^2_+\,V^*)\simeq\RPt$  given by the equations $\beta\wedge\beta=0,$ $*\beta=\beta.$ Similarly for the ASD 2-planes $\T^-V$.  



A {\em null subspace}  is a subspace of $V$ on which the quadratic form $\<\,, \>$ vanishes. The maximum dimension of a null subspace is 2, in which case we call it a {\em null 2-plane}. It turns out that the null 2-planes are precisely the SD and ASD 2-planes. 



\begin{proposition}\label{Nine} Let $V$ be an oriented 4-dimensional vector space  equipped with a quadratic form of signature $(2,2)$. Then 
\begin{enumerate}[leftmargin=*]\setlength\itemsep{5pt}
\item A 2-plane  $W\subset V$ is null if and only if it is  SD or ASD. Thus the space $Gr_0(2, V)$ of null 2-planes in $V$ is naturally identified with 
$$Gr_0(2,V)=(\T^+V)\sqcup(\T^- V),\quad \T^\pm V\simeq \RP^1.$$
\item 
Every  1-dimensional null subspace $N\subset V$  is the intersection of  precisely two null 2-planes, one SD and one ASD, $N=W^+\cap W^-.$  

\end{enumerate}
\end{proposition}

The proof is elementary (omitted). Let us just describe briefly the  picture that emerges from   the last assertion. The set of 1-dimensional null subspaces $N\subset V$  forms  the {\em projectivized null cone}  $\P C$,  a 2-dimensional quadric surface in $\P V\simeq\RP^3$, given in homogeneous coordinates, with respect to a null basis in $V$, by the equation $x^ax_a=0$. 
The statement then is that the SD and ASD null 2-planes in 
$V$ define a  {\em double ruling} of $\P C$. That is, the surface $\P C\subset \P V$, although not flat, contains many lines, forming a pair of foliations, so that through each point $e\in\P C$ pass exactly two lines, one from each foliation. The two lines through $e$ can also be found by intersecting $\P C$ with  the tangent plane to  $\P C$ at $e$. In some  affine chart, if $\P C$ is given by  $z=xy$ and  $e=(x_0,y_0, x_0y_0)$, then  the two null lines through $e$ are given by $z=x_0y$, $z=xy_0$. 
\begin{figure}[H]\centering
\includegraphics[width=0.3\textwidth]{double_ruling5}
\caption{The double-ruling of the projectivized null cone  $\P C\subset \P V$.}
\end{figure}



\subsubsection{The Levi-Civita connection and its curvature}\label{LC}


Now let $M$ be an oriented  smooth 4-manifold equipped with a pseudo-riemannian metric  $\met$ of signature $(2,2)$. 
Denote by $\BLA^k:=\Lambda^k(T^*M)$ the bundle of differential $k$-forms on $M$ and by  $\Gamma(\BLA^k)$ 
its space of smooth sections. In a (local) null coframe $\eta=(\eta^1, \eta^2, \eta_1, \eta_2)^t\in \Gamma(\BLA^1\otimes\R^4)$
the metric is  given by $\met=2\eta_a\eta^a$ and the Levi-Civita connection  is given by the unique 
 $\so_{2,2}$-valued  1-form $\Theta$ satisfying $d\eta+\Theta\wedge\eta=0$, i.e.  the connection is {\em torsion-free}. The associated covariant derivative is $\nabla \eta=-\Theta\otimes \eta$ 
and the curvature is the $\so_{2,2}$-valued 2-form  $\Phi=d\Theta+\Theta\wedge\Theta.$ The curvature form  $\Phi$ defines  via the isomorphism $\so_{2,2}\simeq \Lambda^2(T^*_mM)$ of  Eqn.~(\ref{iso2}) the curvature   {\em operator} $\cR\in\Gamma(\End(\BLA^2))$, which is  self-adjoint with respect to $\met$, i.e. $\cR^*=\cR$. Now  we use the  decomposition $\BLA^2=\BLA^2_+\oplus\BLA^2_-$ to    block decompose
\be\label{decompo}
\cR=\left(\begin{array}{cc}
\cA^+&\cB\\ 
\cB^*& \cA^-
\end{array}\right),\ee
where $\cB\in \mathrm{Hom}(\BLA^2_+, \BLA^2_-)$ and
$\cA^\pm\in\End\,\BLA^2_\pm$ are self-adjoint. This can be further refined into an  irreducible decomposition
$$\hbox{$\cR\sim  (\tr\cA^\pm,\,\cB,\,\cA^+-{1\over 3}\tr\cA_+, \,\cA^--{1\over 3}\tr\cA^-)$},$$
where $\tr\cA^+ =\tr\cA^- ={1\over 4}$ scalar curvature, $\cB$ is the traceless Ricci tensor and the last two components are traceless endomorphisms $\cW^\pm\in \Gamma(\End_0(\BLA^2_\pm))$,  defining the conformally invariant {\em Weyl tensor}, $\cW:= \cW^+\oplus\cW^-$  \cite{ST}. Thus the metric is {\em Einstein} iff $\cB = 0,$ {\em conformally flat} iff $\cW=0$, {\em self-dual} iff $\cW=\cW^+$ (i.e. $\cW^-=0$) and {\em anti-self-dual} iff $\cW=\cW^-$ (i.e. $\cW^+=0$). 

\subsubsection{Principal null 2-planes} Associated with  the Weyl tensor $\cW$ are  its {\em principal null 2-planes}, as follows. 
Recall  from Sect.~\ref{linalg} (just before Prop.~\ref{Nine}) that a 2-plane $ W\subset T_mM$  corresponds  to  a unique 1-dimensional space $\R\beta\subset \Lambda^2(T^*_mM)$ satisfying $\beta\wedge\beta=0$;  also,   $W$ is SD iff $\beta\in\BLA^2_+$, ASD iff $\beta\in\BLA^2_-$. 
\begin{definition}\label{def_principal}
A null 2-plane $W\subset  T_mM$  is {\em principal} if the associated  non-zero elements $\beta \in \Lambda^2(T^*_mM)$ satisfy $\beta\wedge\cW\beta=0.$ 

 \end{definition}
%
 If $\cW^+_m=0$ then  all SD null 2-plane in $T_mM$ are principal (by definition). Otherwise, the quadratic equation  $\beta\wedge\cW^+\beta=0$ defines a conic in $\P\Lambda^2_+(T^*_mM)\simeq \RP^2$, intersecting the conic $\T^+(T_mM)$ given by $\beta\wedge\beta=0$ in at most 4 points, corresponding precisely to  the principal SD 2-planes. The possible patterns of intersection  of these two conics  give rise to an algebraic classification of the SD Weyl tensor $\cW^+$, called the 
 {\em Petrov classification}. A similar classification holds for  $\cW^-$. 
\begin{figure}[h]
\centering
\includegraphics[width=1\textwidth]{petrov1}
\caption{The Petrov classification}\label{petrov}
\end{figure} 
 
\begin{rmrk} The above diagram depicts the classification over $\C$.  In the real case (such is ours) there are more sub-cases, as some of the intersection points might be complex. See for example \cite{GHN} for the complete classification. 
\end{rmrk} 

\subsubsection{The twistor fibration and distribution}\label{twist}

(We shall state the  results for the SD twistor fibration, but they apply verbatim  to the ASD case as well). 
Let $M$ be an oriented 4-manifold with a split-signature pseudo-riemannian metric, as in the previous subsection. The  {\em SD (self-dual) twistor fibration} is the fibre bundle  $$\RP^1\to\T^+M\to M$$  
 whose fiber at a point $m\in M$ is the set  $\T^+(T_mM)$ of  SD null 2-planes in $T_m M$ (see Prop.~\ref{Nine} of Sect.~\ref{linalg} above). 
 The total space $\T^+M$  is a 5-manifold  equipped with a natural rank 2 distribution $\D^+\subset T(\T^+M)$, the {\em SD twistor distribution},  defined by the Levi-Civita connection, as follows: a point $\tilde m\in\T^+_mM$  corresponds to a SD 2-plane $W\subset T_mM$; the 2-plane $\D^+_{\tilde m}\subset T_{\tilde m}(\T ^+M)$ is the horizontal lift of $W$  via  the Levi-Civita connection (one can check that $\D^+$ depends only on the conformal class $[\met]$  of the metric on $M$). By construction, the integral curves of $\D^+$ project to null-curves in $M$ with parallel self-dual tangent  2-plane. Conversely, each null curve in $M$ with parallel SD null 2-plane lifts uniquely to an integral curve of $(\T^+M, \D^+).$
 
   This  is the split-signature version  of the famous twistor construction of Roger Penrose \cite{Pen}.
A  standard feature  of  the twistor construction is  the relation between the integrability properties of $\D^+$ and  the vanishing of the SD Weyl  tensor $\mathcal W^+$. Namely,   $\D^+$ is integrable iff $\cW^+\equiv 0$ (i.e. $M$ is ASD). Less standard is the case of non-vanishing $\cW^+$, treated by An-Nurowski in \cite{AN}. 


 \begin{theorem}[\cite{AN}]
 Let $(\T^+M,\D^+)$ be the SD twistor space and distribution of a split-signature oriented pseudo-riemannian conformal 4-manifold $(M,[\met])$ with a nowhere-vanishing SD Weyl tensor $\cW^+$. Then $\D^+$ is  $(2,3,5)$  away from the principal locus of  $\T^+M$. That is,  $\D^+$ is  $(2,3,5)$ when restricted to the open subset  $\T^+_*M\subset \T^+M$ obtained by removing the set of points corresponding to the principal SD 2-planes (at most 4 points on each fiber of $\T^+M\to M$; see Def.~\ref{def_principal} above).
 \end{theorem} 
See the theorem   in \cite{AN}, right before Corollary 1. 

 

\subsection{The tangent SD 2-plane along a null curve in the dancing space.}\label{tang_SD}
Now we return to our case of $\M\subset \RPt\times \RPts$ equipped with the dancing metric $\met$, as defined in Prop.~\ref{dancing}.

\begin{definition}\label{ng}
Let $\Gamma$  be a parametrized curve in $\M$,  $\Gamma(t)=(q(t),p(t)) $. 
Then  $\Gamma $ is  {\em non-degenerate} if $q(t),p(t)$ are regular curves in $\RPt, \RPts$ (resp.); i.e., $q'(t)\neq 0$ and $ p'(t)\neq 0$ for all $t$. 
\end{definition}

Note that the non-degeneracy condition  is reparametrization independent, hence it applies to unparametrized curves $\Gamma\subset M$ (1-dimensonal submanifolds). It means that $\Gamma$ 
is nowhere tangent to the leaves of the double fibration $\RPt\leftarrow\M\to\RPts$.   
Equivalently, the projections of $\Gamma$ to $\RPt$ and $\RPts$ are non-singular. 




 Now let $\Gamma$ be a null-curve in $ (\M,[\met])$. Then, by Prop. \ref{Nine}, there are two tangent null 2-plane fields defined along $\Gamma$, one SD and the other ASD, whose  intersection is the tangent line field along $\Gamma$. 
\begin{theorem}\label{eleven}
Every  integral curve $\tG$  of $(\Q,\D)$ projects to a non-degenerate  null-curve $\Gamma$  in  $\M$ with a parallel SD tangent 2-plane. Conversely, every non-degenerate null-curve in $(\M, [\met])$ with parallel SD tangent 2-plane lifts uniquely to an  integral curve  of $(\Q,\D)$. \end{theorem}


\begin{theorem}\label{ident} For each $(\q,\p)\in\Q$, the 2-plane 
$$\Pi_*\D_{(\q,\p)}\subset T_{(q,p)}\M,$$
where $(q,p)=\Pi((\q, \p)),$
is a non-principal self-dual 2-plane. The resulting map 
$$ \Q\to \T^+\M,\quad (\q,\p)\mapsto \Pi_*\D_{(\q,\p)},$$ is an  $\SLth$-equivariant  embedding, identifying    $\Q$ with  the non-principal locus of $\D^+$ in $\T^+\M$, and mapping $\D$ over to $\D^+$. 
\end{theorem}

The proofs of these two theorems will be carried out in the next subsection, using  the Maurer-Cartan structure equations of $\SLth$. 




\subsection{Proofs of Theorems~\ref{eleven} and \ref{ident}}\label{proofs}

 Let $\{e_1,e_2,e_3\}$ be the standard basis of $\Rt$ and $\{e^1,e^2,e^3\}$ the dual basis of $\Rts$. Recall (Sect.~\ref{TwoFour}) that  $G=\SLth$ acts transitively on $(\Q,\D)$ and $(\M,\met)$ by $g\cdot(\q, \p)=(g\q, \p g^{-1}),$ $g\cdot([\q], [\p])=([g\q], [\p g^{-1}])$,  preserving  $\D$ and $\met$ (resp.). 
Fix $\tilde m_0=(e_3, e^3)\in \Q$ and  $m_0=\Pi(\tilde m_0)=([e_3], [e^3])\in\M$. Define
\be\label{principal}
    \xymatrix{
        G  \ar[dr]_j \ar[r]^\tj & \Q \ar[d]^\Pi\\
                               & \M }
\ee
%
by $\tj(g)=g\cdot \tilde m_0=(ge_3, e^3g^{-1})$, $ j(g)=g\cdot m_0=([ge_3],[e^3g^{-1}])=(\Pi\circ\tj)(g).$ Then  $j$ is a principal $H$-fibration and $\tj$ a principal $H_0$-fibration, 
where 
$$H=\left\{ \left(
\begin{array}{cc}
A&0\\
0 &
a^{-1}
\end{array}\right)\st \; A\in \GL_2(\R), \; a=\det(A)\right\}
\simeq\GL_2(\R)$$
is the stabilizer subgroup of $m_0$, with Lie algebra
\be\label{lieh}\h=\left\{ \left(
\begin{array}{cc}
X&0\\
0 &
-x
\end{array}\right)\st \; X\in\gl_2(\R), \; x=\tr(X)\right\}\simeq\gl_2(\R)
\ee
and 
$$H_0=\left\{ \left(
\begin{array}{cc}
A&0\\
0 &
1
\end{array}\right)\st \; A\in \SL_2(\R)\right\}
\simeq\SL_2(\R)$$
is the stabilizer subgroup of $\tilde m_0$, with Lie algebra
$$\h_0=\left\{ \left(
\begin{array}{cc}
X&0\\
0 &
0
\end{array}\right)\st \; X\in \sl_2(\R)\right\}\simeq\sl_2(\R).$$

 The left-invariant MC (Maurer-Cartan) form on $G=\SLth$ is the $\g$-valued 1-form  
$\omega=(\om{i}{j}):=g^{-1}dg$, i.e. $\tr(\omega)=\om{i}{i}=0$,  $i,j\in\{1,2,3\}$ (using, as always,  the summation convention on repeated indices). The components of   $\omega$ provide a global coframing on $G$, whose basic properties (immediate from its definition) are
\be\label{MC} \begin{array}{ll}
(a)\quad \omega_e=id_\g&\\
(b)\quad (L_g)^*\omega=\omega & \hbox{(left invariance)}\\
(c)\quad (R_g)^*\omega=g^{-1}\omega g & \hbox{(right $\Ad$-equivariance)}\\
(d)\quad d\omega =-\omega \wedge\omega& \hbox{(the MC structure  equation).}
\end{array}
\ee






Now let us rename  the components of $\omega$: 
\be\label{many}
\eta^a:=\om{a}{3}, \; \eta_b:=\om{3}{b}, \;\phi:=\om{a}{a}=-\om{3}{3},\;\theta^{a}_{\,b}:=\om{a}{b}+\delta^a_{\,b}\phi, \quad a,b\in\{1,2\}.
\ee


Furthermore, introduce the matrix notation
\be\label{compon}
\eta:=
\left(\begin{matrix}\eta^1\\ \eta^2\\ \eta_1\\\eta_2\end{matrix}\right), 
\Theta:=
\left(\begin{array}{cc}
\theta&0\\
0&-\theta^t
\end{array}\right) , 
\theta:=(\theta^{a}_{\,b})=
\left(\begin{array}{cc}
2\om{1}{1}+ \om{2}{2}&\om{1}{2}\\
\om{2}{1}&\om{1}{1}+ 2\om{2}{2}
\end{array}\right).
\ee
%
With this notation, Eqn.~(\ref{MC}c) now  reads
\be\label{comp}(R_h)^*\eta=\rh^{-1}\eta,\quad (R_h)^*\Theta=\rh^{-1}\Theta\rh,\quad h\in H,
\ee
where $\rho:H\to \SO_{2,2}$ is the {\em isotropy representation},   
\be\label{isotropy}h=\left(
\begin{array}{cl}
A&0\\ 0& a^{-1}
\end{array}
\right)
\mapsto 
\rh=\left(
\begin{array}{cc}
aA&0\\ 0& (aA^t)^{-1}
\end{array}
\right),\quad A\in\GL_2(\R), \quad a=\det A.
\ee
%
The MC structure  equation (Eqn.~(\ref{MC}d)) also  breaks into two equations,
\be\label{break}
d\eta+\Theta\wedge\eta=0,\qquad
d\Theta+\Theta\wedge\Theta=\left(\begin{array}{cc}\varphi&0\\ 0&-\varphi^t\end{array}\right), 
\ee
where
\be\label{curve}
\varphi:=d\theta+\theta\wedge\theta=
\left(\begin{array}{cc}
2\eta_1\wedge\eta^1+ \eta_2\wedge\eta^2&\eta_2\wedge\eta^1\\[5pt]
\eta_1\wedge\eta^2&\eta_1\wedge\eta^1+ 2\eta_2\wedge\eta^2
\end{array}\right).
\ee



From Formula~(\ref{lieh}) for $\h$, we see that the  four 1-forms $\eta^a, \eta_b\in\Omega^1(G)$
 are pointwise linearly independent and $j$-horizontal, i.e. vanish  on the fibers of 
 $j:G\to M$, hence span $j^*(T^*M)\subset T^*(G).$ Similarly, $\eta^a, \eta_b, \phi$ span  $\tj^*(T^*\Q).$
 
 

\begin{proposition}\label{all}    

Consider the principal  fibrations $j, \tj$ of Eqn.~\eqref{principal} and the left-invariant 1-forms $\eta, \phi, \theta, \Theta, \varphi$ on $G$, as defined above in Eqns.~\eqref{many}-\eqref{curve}. 

\begin{enumerate}[leftmargin=18pt,label=(\alph*)]\setlength\itemsep{5pt}
\item  $j^*\met=2\eta_a\eta^a$, where $\met$ is the dancing metric on $\M$, as defined in Prop.~\ref{dancing}. 

\item Let $\nabla$ be the covariant derivative on $T^*M$ associated with   the Levi-Civita connection of   $\met$ and $\wnab=j^*(\nabla)$ its pull-back to $j^*(T^*M)$.Then $\wnab\eta^a=-\theta^a_{\,b}\otimes\eta^b, $ $\wnab\eta_b=\theta^a_{\,b}\otimes\eta_a,$ or in matrix form,
$\wnab\eta=-\Theta\otimes\eta$. The associated  curvature 2-form   is $\Phi:=d\Theta+\Theta\wedge\Theta$,  given in terms of $\eta$ by   Eqns.~\eqref{break})-\eqref{curve} above. 


 
\item  Let $vol\in\Omega^4(M)$ be the positively oriented unit volume form on $\M$ (see  Sect.~\ref{orientation}). Then  $j^*(vol)=\eta^1\wedge\eta^2\wedge\eta_1\wedge\eta_2.$ 


\item   Let  $\D\subset T\Q$ be the rank 2 distribution given by $d\p=\q\times d\q$ and $\D^0\subset T^*\Q$ its annihilator.  
Then   $\tj^*(\D^0)=\Span\{\eta^2-\eta_1,\;\eta^1+\eta_2, \; \phi\}.$

\end{enumerate}
\end{proposition}

\newcommand{\hT}{\hat\Theta}
\begin{rmrk} We can rephrase the above in terms of coframes on $M$
 and $Q$, as follows: let $\sigma$ be a local section of $j:G\to M$, 
 then (a) $\weta=\sigma^*\eta$ is a null-coframe on $M$, 
 so that $\met=2\weta_a\weta^a$, 
 (b)  $\hT=\sigma^*(\Theta)$ is the connection 1-form of the 
 Levi-Civita connection of $\met$ with respect to the coframe $\weta$, 
 (c) $vol=\weta^1\wedge\weta^2\wedge\weta_1\wedge\weta_2.$ and
  (d) $\D=\Ker\{\tilde\eta^2-\tilde\eta_1,\;\tilde\eta^1+\tilde\eta_2, \; \tilde\phi\},$
   where $\tilde\eta=\tilde\sigma^*\eta,$ $\tilde\phi=\tilde\sigma^*\phi $ and $\tilde\sigma= \sigma\circ\Pi$ (a local section of $\tj:G\to Q$). 
\end{rmrk}
\begin{proof} 
 (a) First,  the formula $(R_h)^*\eta=\rh^{-1}\eta$ of Eqn.~(\ref{comp}) implies  that $\eta_a\eta^a$, a $G$-left-invariant $j$-horizontal symmetric 2-form   
 on $G$, is  $H$-right-invariant, hence descends to a 
well-defined $G$-invariant symmetric 2-form on $M.$ Next, by examining the isotropy
 representation of $H$ (Eqn.~(\ref{isotropy})), one sees that $T_{m_0}M$ admits  a unique 
 $H$-invariant quadratic form, up to a constant multiple, hence $M$ admits a unique $G$-invariant 2-form, up to a constant multiple.  It follows that it is enough to verify the equation $j^*\met=2\eta_a\eta^a$ 
 on a single non-null element $Y\in \g=T_eG$; for example,  $Y=Y_1$ from  the proof of Prop.~\ref{prop235}. We omit this (easy) verification.
  
\mn(b) The relations   $d\eta+\Theta\wedge\eta=0$, $(R_h)^*\Theta=\rh^{-1}\Theta\rh$  and the formula for $\Theta$
(Eqns.~(\ref{compon})-(\ref{break}))  show  that $\Theta$ is an $\so_{2,2}$-valued 1-form on $G$, descending to  a torsion-free $\SO_{2,2}$-connection  on $T^*M$, hence  is in fact the Levi-Civita connection of $\met$. 


\mn(c)  First one verifies that $\eta^1\wedge\eta^2\wedge\eta_1\wedge\eta_2$ is  a volume  form of norm 1 with respect to $2\eta_a\eta^a$. Then, to compare to the orientation definition of Sect.~\ref{orientation}, we check that $K^*\eta^a=\eta^a,$ $K^*\eta_b=-\eta_b,$ hence $\eta^1+\eta_1, \eta^2+\eta_2, \eta^1-\eta_1, \eta^2-\eta_2$ is a para-complex coframe. Now one calculates  $(\eta^1+\eta_1)\wedge(\eta^2+\eta_2)\wedge(\eta^1-\eta_1)\wedge(\eta^2-\eta_2)=4\eta^1\wedge\eta^2\wedge\eta_1\wedge\eta_2,$ hence $\eta^1, \eta^2, \eta_1, \eta_2$ is a positively oriented coframe. 


\mn(d)  Let $E_j:G\to \Rt$ be the function that assigns  to an element $g\in G$ its   $j$-th  column, $j=1,2,3$. Then $\omega=g^{-1}dg$ is equivalent to $dE_j=E_i\om{i}{j}$. Next let $E^i:G\to \Rts$ be the function  assigning to $g\in G$ 
the $i$-th row of $g^{-1}$. Then clearly $E^iE_j=\delta^i_{\,j}$ (matrix multiplication of a row by column vector), and by taking exterior derivative of the last equation  we obtain $dE^i=-\om{i}{j}E^j.$ Also, $\det(g)= 1$ implies $E_i\times E_j=\epsilon_{ijk}E^k$, $E^i\times E^j=\epsilon^{ijk}E_k$. Next, by definition of $\tj$, $E_3=\q\circ\tj,$ $E^3=\p\circ\tj$. Now we calculate 
$\tj^*(d\p-\q\times d\q)=dE^3-E_3\times dE_3=-\om{3}{j}E^j-(E_3\times E_i)\om{i}{3}=(\eta^2-\eta_1)E^1-(\eta^1+\eta_2)E^2+\phi E^3.$
\end{proof}




\begin{cor}[Proofs of Thms.~\ref{eleven} and \ref{ident}]\label{grancor}
\n\begin{enumerate}[leftmargin=18pt,label=(\alph*)]\setlength\itemsep{5pt}
\item $(\M,\met)$ is Einstein but not Ricci-flat, SD (i.e. $\cW^-\equiv 0$) and $\cW^+$ is nowhere vanishing, of Petrov type D (see 
Fig.~\ref{petrov}). 
More precisely,  at each $(q,p)\in M$ there are exactly two principal SD null 2-planes, 
each of multiplicity 2, given by $T_q\RPt\oplus \{0\}$ and $\{0\}\oplus T_p\RPts. $ 




\item Every integral curve $\tG$ of $(Q,\D)$ projects to a non-degenerate null curve $\Gamma:=\Pi\circ\tG$ in  $(\M,[\met])$ with parallel SD tangent 2-plane. 

\item Every non-degenerate null curve $\Gamma$ in $(\M,[\met])$ with parallel SD tangent 2-plane lifts uniquely to an integral curve $\tG$ of $(\Q, \D).$


\item For every $\tilde m\in \Q$,  $\Pi_*\D_{\tilde m}\subset T_{\Pi(\tilde m)}\M$ is a non-principal SD null 2-plane.  


\item Let $\T^+_*M\subset \T^+M$ be the non-prinicipal locus (the complement of the principal points). Then the map $\nu:\Q\to \T^+_*M$, 
$\tilde m\mapsto \Pi_*\D_{\tilde m}$, 
is  an $\SLth$-equivariant diffeomorphism, mapping $\D$ unto $\D^+$. 



\end{enumerate}

\end{cor}




\begin{proof}

\mn (a)   By Prop.~\ref{all}a and \ref{all}c, the coframe $\eta^1, \eta^2,\eta_1, \eta_2$ is null and positively oriented. It follows from the definition of the Hodge dual  that
%
%
\begin{subequations}
\begin{align}\label{iso}
\quad j^*( \Lambda^2_+\,M)&= 
\Span\{ \eta_1\wedge\eta^1+\eta_2\wedge\eta^2, \;
\eta^1\wedge\eta^2, \; 
\eta_1\wedge\eta_2
\},\\
%\\
\quad j^*( \Lambda^2_-\,M)&= 
\Span\{ \eta_1\wedge\eta^1-\eta_2\wedge\eta^2, \;
\eta_1\wedge\eta^2, \; 
\eta_2\wedge\eta^1
\}.
\end{align}
\end{subequations}



Then using  the formula for the curvature form $\Phi$ (Eqns.~(\ref{break})-(\ref{curve})) and the definition of the curvature operator  $\cR$
(Sect.~\ref{LC}), one finds that $j^*(\cR)$ is diagonal in the above bases,  with matrix 
$$
j^*(\cR)={\footnotesize\left(\begin{array}{ccc|ccc}
-3&&&&&\\ 
&\;0&&&&\\ 
&&\;0&&&\\
\hline
&&&-1&&\\ 
&&&&-1&\\ &&&&&-1
\end{array}\right)}.
$$ 
Comparing this expression with the the decomposition of $\cR$ of Eqn.~(\ref{decompo}), we see that the dancing metric is Einstein ($\cB\equiv 0$), the scalar curvature is  $-12$,  $\cW^-\equiv 0$ and 
$$j^*(\cW^+)={\footnotesize\left(
\begin{matrix}-2&&\\ &1&\\ &&1\end{matrix}\right)}.
$$
Now let  $a,b,c$  be the  coordinates dual to the basis of  $ j^*( \Lambda^2_+\,M)$  of Eqn.~(\ref{iso}). Then $\beta\wedge\beta=0$
 is given by $a^2-bc=0$ and $\beta\wedge\cW^+\beta=0$  by $2a^2+bc=0$. This system of  two homogeneous equations has two non-zero
  solutions (up to a non-zero multiple), $a=b=0$ and $a=c=0$, each with multiplicity 2 (the  pair of conics defined in
  each fiber of $\P\Lambda^2_+M$ by these equations are tangent at their two  intersection points). 
  The corresponding SD  2-forms are $\eta_1\wedge\eta_2, \eta^1\wedge\eta^2$,  corresponding  
  to the principal SD null 2-planes  $T_q\RPt\oplus \{0\}$ and $\{0\}\oplus T_p\RPts $ (resp.), as claimed. 

\mn(b)  Let $\tG(t)=(\q(t),\p(t))$  be  a regular parametrization of an integral curve of $(\Q, \D)$, i.e.  $\tG'=(\q', \p')$  is nowhere vanishing and  $\p'=\q\times \q'$. We  first show that $\Gamma=\Pi\circ\tG$ 
  is non-degenerate. Let $\Gamma(t)=\Pi(\tG(t))=(q(t), p(t)),$ 
  where $q(t)=[\q(t)],$ 
  $p(t)=[\p(t)].$ We need to show that 
 $q',  p'$ are nowhere vanishing.


\begin{lemma}\label{dual} The distribution $\D\subset T\Q$, given by $d\p=\q\times d\q$, is also given by $d\q=-\p\times d\p.$ 
\end{lemma}
\begin{proof} Let  $\D'=\Ker(d\q+\p\times d\p)\subset T\Q$. Then both $\D,\D'$ are $\SLth$-invariant, hence it is enough to compare them at say $(e_3, e^3)\in\Q$. At this point $\D$ is given by $dp_1+dq^2=dp_2-dq^1=dp_3=dp^3+dq_3=0$, and $\D'$ by $dq^1-dp_2=dq^2+dp_1=dq^3=dp^3+dq_3=0.$ These obviously have the same 2-dimensional space of  solutions. 
\end{proof}

Now   $q'=\q'\,(\mod \q)$, hence  $q'=0\ent  \q'\times\q=0\ent \p'=0,$ so by Lemma \ref{dual}, $\q'=-\p\times\p'=0$. Similarly, $p'=0\ent \q'=\p'=0,$ hence $\Gamma$ is non-degenerate. 


 Next we show that $\Gamma$ is null. Let $\sigma$ be a lift of $\tG$ (hence of  $\Gamma$) to $G=\SLth$. Let 
$\sigma^*\eta^a=s^adt, $ $\sigma^*\eta_b=s_bdt,$ $a,b=1,2, $ for some  real-valued functions (of $t$) $s^1, s^2, s_1, s_2$.  Then, by Prop.~\ref{all}a and \ref{all}d, $\met(\Gamma', \Gamma')=2s_as^a=
2(s_1 (-s_2)+s_2s_1)=0,$ hence $\Gamma$ is a null curve. 

 Next we show that  the SD null 2-plane along $\Gamma$  is parallel.
Let  $\weta=\eta\circ\sigma$ be the  coframing of $\Gamma^*(TM)$ determined by the lift $\sigma$ of $\Gamma$ (a ``moving coframe"  along 
$\Gamma$).

\begin{rmrk} $\weta$  should not be  confused with
 $\sigma^*\eta=(s^1, s^2, s_1, s_2)^tdt$, the restriction of $\weta$ to $T\Gamma$.   
 \end{rmrk}
 
 Let  $W$ be  the 2-plane field along 
 $\Gamma$ defined by  $\weta^1+\weta_2=\weta^2-\weta_1=0$.  By Prop.~\ref{all}d, 
 $\sigma^*(\eta^1+\eta_2)=\sigma^*(\eta^2-\eta_1)=0,$ hence $W$ is tangent to
 $\Gamma$. The  2-form corresponding to $W$ is 
 $\beta=(\weta^1+\weta_2)\wedge(\weta^2-\weta_1)=
 \weta^1\wedge\weta^2+\weta_1\wedge\weta_2+\weta_1\wedge\weta^1+\weta_2\wedge\weta^2$, 
 which is SD by  Formula~(\ref{iso}),  hence $W$ is the  SD tangent 2-plane field along $\Gamma$.
  Now a short calculation, using Prop.~\ref{all}b, shows that
  $$\nabla \beta =3(\sigma^*\phi)\otimes (\weta_1\wedge\weta_2-\weta^1\wedge\weta^2).$$ By Prop.~\ref{all}d, $\sigma^*\phi=0$, hence $\nabla \beta=0$, so $W$ is parallel. 

\mn(c)  Let $\sigma$ be a lift of $\Gamma$ to $G$, with 
$\sigma^*\eta^a=s^adt, $ $\sigma^*\eta_b=s_bdt.$ 

\begin{lemma}\label{adapted} Given a non-degenerate parametrized null curve $\Gamma:\R\to M$, there exists a lift $\sigma$ of $\Gamma$ to $G$ such that  $s^1=s_2=0,$ $s_1=s^2=1$. In other words, 
$$\sigma^*\omega=%{\footnotesize 
\left(\begin{array}{ccc} *&*&0\\ *&*&1\\ 1&0&*\end{array}\right)dt.
%}
$$
\end{lemma}
 
\begin{rmrk} We call such a lift $\sigma$ {\em adapted} to $\Gamma$.  
\end{rmrk}
 
\begin{proof} Starting with an  arbitrary lift $\sigma$, any other lift is of the form 
$\bar\sigma=\sigma h$, where $h:\R\to H$ is an arbitrary $H$-valued smooth function, i.e. 
$$h= %{\footnotesize 
\left(
\begin{array}{cl}
A&0\\
0 &
a^{-1}
\end{array}\right), \quad  A:\R\to \GL_2(\R), \; a=\det(A).
%}
$$
%
 Now a short calculation shows that 
 $$\bar\sigma^*\omega=
 h^{-1}(\sigma^*\omega)h+h^{-1}dh=
 %{\footnotesize 
\left(\begin{array}{ccc} *&*&0\\ *&*&1\\ 1&0&*\end{array}\right)dt
%}
$$ 
provided 
\be\label{B}
aA \left(\begin{matrix} 0\\  1\end{matrix}\right)=
 \left(\begin{matrix}  s^1\\ s^2\end{matrix}\right), \quad 
(1, \; 0)=
( s_1, \;  s_2)aA.
%}
\ee
Now one checks that the  last system of equations can be solved for $A$  iff  $(s^1,  s^2)\neq 0,$ $(s_1, s_2)\neq 0$ 
and $s_as^a=0$. These are precisely the non-degeneracy and nullity conditions on $\Gamma$. From  $A$ we obtain  $h$ and  the desired $\bar\sigma$.
\end{proof}
%
Once we have an adapted lift $\sigma$ of $\Gamma$, with associated moving coframe $\weta:=\eta\circ\sigma$, we define a 2-plane field $W$  along $\Gamma$ 
  by   
 $\weta^1+\weta_2=\weta^2-\weta_1=0$. Then 
 $\sigma^*(\eta^1+\eta_2)=(s^1+s_2)dt=0,$ $\sigma^*(\eta^2-\eta_1)=(s^2-s_1)dt=0,$ hence $W$ is tangent to $\Gamma$. Let $\beta:=(\weta^1+\weta_2)\wedge(\weta^2-\weta_1).$ 
 Then $\beta$ is SD, so  $W$ is the SD tangent 2-plane along 
 $\Gamma$. Now  $W$ is  parallel  
 $\ent \nabla \beta =3(\sigma^*\phi)\otimes (\weta_1\wedge\weta_2-\weta^1\wedge\weta^2)\equiv 0\,(\mod \beta)\ent \sigma^*\phi=0$, since $\weta_1\wedge\weta_2-\weta^1\wedge\weta^2$ is a non-zero ASD  form, hence  $\not\equiv 0\,(\mod\beta).$ It follows that $\sigma$ satisfies $\sigma^*(\eta^1+\eta_2)=
\sigma^*(\eta^2-\eta_1)= \sigma^*(\phi)=0$, hence, by Prop.~\ref{all}d, $\tG:=\tj\circ\sigma$ is a lift of $\Gamma$ to an integral curve of $(Q, \D)$. 



To show uniqueness, if $\tG(t)=(\q(t), \p(t))$ then  any other lift of $\Gamma$ to $\Q$ is of the form $(\lambda(t)\q(t), \p(t)/\lambda(t))$ for some non-vanishing real function $\lambda(t)$. If this other lift is also an integral curve of $(\Q, \D)$ then 
$(\p/\lambda)'-(\lambda\q)\times( \lambda\q)'=-(\lambda'/\lambda^2)\p+(1/\lambda-\lambda^2)\p'=0.$ 
Multiplying the last equation by $\q$ and using $\p\q=1, \p'\q=0$, we get $\lambda'=0\ent (1-\lambda^3)\p'=0$. Now $\p'\neq 0$ since $\Gamma$ is non-degenerate $\ent \lambda^3=1\ent \lambda= 1.$
 
 
 
\mn(d)  Let $m=\Pi(\tilde m)$,  $W=\Pi_*(\D_{\tilde m})\subset T_mM$, $g\in G$ such that $\tj(g)=\tilde m$  and  $\weta=\eta(g)$ the corresponding coframing of $T_mM$. Then, by Prop.~\ref{all}d,  $W=\Ker\{\weta^1+\weta_2, \weta^2-\weta_1\}.$ As before (item (b)),  one checks that $\beta:=(\weta^1+\weta_2)\wedge(\weta^2-\weta_1)=\weta^1\wedge\weta^2+\weta_1\wedge\weta_2+\weta_1\wedge\weta^1+\weta_2\wedge\weta^2$ is SD $\ent W$ is SD, but not principal (the SD 2-planes are given by $\weta^1\wedge\weta^2$ and $\weta_1\wedge\weta_2$; see the proof of item (a) above). 

 
  
\mn(e)  One checks that $\nu$ is  $\SLth$-equivariant, $\Q$ and $\T^+_*M$ are $\SLth$-homogeneous manifolds, with the same stabilizer at $\tilde m_0=(e_3, e^3)$ and $\nu(\tilde m_0)$, hence $\nu$  is a diffeomorphism. It remains to show that   $\nu_*\D=\D^+$. This is just a reformulation  of items (b) and (c) above. 
\end{proof}




