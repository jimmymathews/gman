
 
\section{Geometric reformulation: a $(2,3,5)$-distribution} 
\subsection{First integral and reduction to the 5-manifold $\Q\subset \Rtt$}\label{TwoOne}
Let $\Rtt:=\R^3\oplus\Rts$, equipped with the quadratic form $(\q,\p)\mapsto \p\q=p^iq_i$. 
One can easily check that  $\p\q$ is  a first integral of equations \eqns\ (a
conserved quantity). That is,  for each $c\in\R$, a solution to \eqns\ that starts on the   level surface $$Q_c=\{(\q,\p)|\p\q=c\}$$ remains on
$Q_c$ for all times.  

Furthermore,  the map
$(\q,\p)\mapsto (\lambda \q, \lambda^2 \p)$, $\lambda\in\R$, maps
solutions on $Q_c$ to solutions on $Q_{\lambda^3c}$, hence it is
enough to study solutions of the system restricted to one of the 
(non-zero) level surfaces, say $$\Q:=\{(\q,\p)|\p\q=1\},$$ a 5-dimensional 
%
affine quadric
%
of signature
$ (3,3).$


\begin{rmrk} An {\em affine 
quadric}  is the non-zero  level set of a non-degenerate quadratic form on $\R^n$; its {\em signature}   is the signature of the defining quadratic form.
\end{rmrk}

\begin{rmrk}  We leave out the less interesting case of the zero level surface $Q_0$. 
\end{rmrk}

Now  restricted to $\Q$,  the equation $\p\q'=0$ is a consequence of $\p'=\q\times\q'$, hence we can replace equations \eqns\ with 
the somewhat simpler system 
\begin{equation}\label{eqnss}
\p\q=1, \quad \p'=\q\times \q'.
\end{equation}

\subsection{A rank 2 distribution $\D$ on $\Q$}\label{TwoTwo}
A geometric reformulation of  Eqns.~\eqref{eqnss} is the following: let us introduce the three 1-forms
$$\omega_i:=dp_i-\epsilon_{ijk}q^jdq^k\in\Omega^1(\Q),\quad i=1,2,3,$$
or in vector notation,
$$\o=d\p-\q\times d\q\in\Omega^1(\Q)\otimes\Rts.$$
%
Then the kernel of the 1-form $\o$ (the common kernel of its 3 components)  
defines  at each point $(\q,\p)\in \Q$ a 2-dimensional linear subspace  
$\D_{(\q,\p)}\subset T_{(\q,\p)}\Q$. Put together, these subspaces define 
a rank 2 distribution  $\D\subset T\Q$  (a field of tangent 2-planes on $\Q$),  
so that the solutions to our system of Eqns.~\eqref{eqnss} are precisely the 
{\em integral curves} of $\D$: the  parametrized curves $(\q(t) , \p(t) )$ 
whose velocity vector at each moment $t$  belongs to $\D_{(\q(t) ,\p(t) )}$.

\begin{proposition} The kernel of $\o=d\p-\q\times d\q$ defines on $\Q$ a rank 2 distribution $\D\subset T\Q$, whose integral curves are given by solutions to Eqns.~\eqref{eqnss}. 
\end{proposition}
\begin{proof} One checks easily that the 3 components of $\o$ are  linearly independent.
\end{proof}

\subsection{$\D$ is a $(2,3,5)$-distribution}  
We recall first   some standard terminology  from the general theory of distributions.
A distribution $\D$ on a manifold  is {\em integrable} if $[\D,\D]\subset\D$. It  is  {\em bracket generating} if one can obtain any tangent vector on the manifold by  successive Lie brackets of vector fields tangent to $\D$. Let $r_i=rank(\D^{(i)})$, where $\D^{(i)}:= [\D, \D^{(i-1)}]$  and $\D^{(1)}:=\D$. Then $(r_1, r_2, \ldots)$,  is the {\em growth
 vector} of $\D$. In general, the growth vector of a distribution may vary from point to point of the manifold, although not in our case, since our distribution is homogeneous, as we  shall soon see.  A distribution with constant growth vector is {\em regular}.  It can be shown that for  a regular  bracket-generating rank 2 distribution on  a 5-manifold there are only two possible growth vectors: $(2,3,4,5)$ (called {\em Goursat distributions}) or $(2,3,5)$, which is the generic case  (see \cite{BrHs}). 

\begin{definition} A {\em $(2,3,5)$-distribution} is  a bracket-generating  rank 2 distribution $\D$  on a 5-manifold $\Q$ with growth vector $(2,3,5)$ everywhere. That is, $\D^{(2)}=[\D,\D]$ is a rank 3 distribution, and 
$\D^{(3)}=[\D,\D^{(2)}]=T\Q$. 
\end{definition}


\begin{proposition} \label{prop235} $\D=\Ker(\o)\subset T\Q$, defined by Eqns.~\eqref{eqnss} above,  is a  $(2,3,5)$-distribution. 
\end{proposition}



 This is a calculation done most  easily  using the symmetries of the equations, so is postponed to the next subsection.



\subsection{$\SLth$-symmetry} \label{TwoFour}
A {\em symmetry} of a distribution $\D$ on a manifold $\Q$  is a diffeomorphism of $\Q$
 which preserves $\D$. An {\em infinitesimal symmetry} of $\D$  is a  vector field on $\Q$ whose flow preserves $\D$. 


The use of the vector and scalar product on  $\R^3$ in Eqns.~\eqref{eqnss} may give the  impression that $\D$ depends on the euclidean structure on $\R^3$, so  $(\Q,\D)$ only admits $\SO(3)$ as  an obvious  group of symmetries (a 3-dimensional group). In fact, it is quite easy to see, as we will show now, that   $(\Q,\D)$   admits $\SLth$  as a symmetry group (8-dimensional). In the next section we will show the less obvious fact that  the symmetry {\em algebra} of $(\Q,\D)$ is $\g_2$ (14-dimensional). 



Fix a volume form on $\R^3$, say $$vol:=dq^1\wedge dq^2\wedge dq^3,$$ and define the associated  covector-valued ``cross-product'' $\R^3\times \R^3\to\Rts$ by 
$$\bv\times \w:=vol(\bv,\w,\cdot),$$
or in coordinates, 
$$(\bv\times \w)_i=\epsilon_{ijk}v^jw^k, \quad i=1,2,3.$$ 











Let $\SLth$ be the group of $3\times 3$ matrices with real entries and determinant 1, acting on  $\Rtt$ by 
\be\label{action}
g\cdot (\q,\p)=(g\q,\p g^{-1})
\ee
 (recall that $\q$ is  a column vector and $\p$ is a row vector). Clearly, this $\SLth$-action leaves the quadratic form $\p\q$ invariant and thus leaves invariant also the quadric $\Q\subset\Rtt$.

Let $e_1,e_2,e_3$ (columns) be the standard basis of $\R^3$ and $e^1, e^2, e^3$ (rows) the dual basis of $\Rts$. 


\begin{proposition} \label{prop:sym} 
 {\em (a)} $\SLth$ acts on $\Q$ transitively and effectively. The stabilizer of $(e_3, e^3)$ is the subgroup 
$$H_0=\left\{\left(\begin{array}{c|c}
A&\begin{matrix} 0\\ 0\end{matrix}\\
\hline
\begin{matrix}0& 0\end{matrix}&1
\end{array}
\right)\st  A\in\SLt\right\}.$$


\mn {\em (b)} $\SLth$ acts on $\Q$ by symmetries of  $\D$. 
\end{proposition}

\begin{proof}
 Part (a) is an easy calculation (omitted). For part (b), note that $\SLth$ leaves $vol$ invariant, hence the vector product $\R^3\times \R^3\to\Rts$ is $\SLth$-equivariant: $(g\bv)\times (g\w)=(\bv\times\w)g^{-1}.$ It follows that $\o=d\p-\q\times d\q$ is also $\SLth$-equivariant, $g^*\o=\o g^{-1}$, hence $\D=\Ker(\o)$ is $\SLth$-invariant.\end{proof}

\begin{proof}[Proof of Prop.~\ref{prop235}] Let $\h_0\subset\sl(3,\R)$ 
be the  Lie algebra of the stabilizer  at $(e_3, e^3)\in \Q$. Pick two 
elements $Y_1, Y_2\in\sl(3,\R)$ whose infinitesimal action at $(e_3, e^3)$ generates $\D$.  Then we need to show that 
$$Y_1, Y_2, [Y_1,Y_2],[Y_1,[Y_1,Y_2]],  [Y_2, [Y_1,Y_2]]$$ span $\sl(3,\R)$ mod $\h_0$ (this will show that $\D$ is $(2,3,5)$ at $(e_3,e^3) $, so by homogeneity everywhere.)
%
Now $\h_0$ consists of matrices of the form 
%
$$
\left(\begin{array}{c|c}
A&\begin{matrix} 0\\ 0\end{matrix}\\
\hline
\begin{matrix}0& 0\end{matrix}&0
\end{array}
\right),\quad A\in\slt.
$$
%
Furthermore, $Y\in\sl(3,\R)$ satisfies $Y\cdot(e_3, e^3)\in\D$ if and only if 
$$Y=\left(\begin{array}{cc}A&\bv\\ \bv^*& 0\end{array}\right),\quad \bv={v_1\choose v_2}, \quad \bv^*=(v_2, -v_1),  \quad A\in\slt.$$ 
We can thus take 
%
$$ 
Y_1=\footnotesize{ 
\left( 
\begin{array}{rrr}
0&0&1\\ 
0&0&0\\ 
0&-1&0
\end{array} 
\right), \quad
Y_2=\left( 
\begin{array}{rrr}
0&0&0\\ 
0&0&1\\ 
1&0&0
\end{array}
\right)},$$
then $[Y_1, Y_2]=Y_3,$ $[Y_1,Y_3]=Y_4$ and $[Y_2,Y_3]=Y_5,$ where 
%
$$ 
Y_3=\footnotesize{ 
\left( 
\begin{array}{rrr}
1&0&0\\ 
0&1&0\\ 
0&0&-2
\end{array}   
\right),  \quad
%
Y_4=
\left( 
\begin{array}{rrr}
0&0&-3\\ 
0&0&0\\ 
0&-3&0
\end{array}  
\right), \quad
Y_5=
\left( 
\begin{array}{rrr}
0&0&0\\ 
0&0&-3\\ 
3&0&0
\end{array} 
\right),}
$$
which together with $Y_1, Y_2$ span $\slth/\h_0\simeq T_{(e_3, e^3)}\Q$.
\end{proof}



%\begin{rmrk} There is a  ``proof without words" of the last proposition, using only the picture of the root diagram of $\slth$. See for example \cite{BM}. 
%\end{rmrk}

%\begin{figure}[h]\centering
%\includegraphics[width=0.4\textwidth]{g2}
%\caption{Proof of Prop.~\ref{prop235} via the  root diagram of $\g_2$.}
%\end{figure}

\subsection{$\g_2$-symmetry via Cartan's submaximality} \label{three_five}Here we show that the symmetry {\em algebra} of our distribution $(\Q,\D)$, given by Eqns.~\eqref{eqnss}, is isomorphic to  $\g_2$, a 14-dimensional simple Lie algebra, the maximum possible for a $(2,3,5)$-distribution. We show this  as an immediate consequence of  a general theorem of Cartan (1910) on $(2,3,5)$-distributions. In the next section this ``hidden symmetry'' is explained and written down explicitly by defining an  embedding  of $(\Q,\D)$ in the standard $\G$-homogeneous model $(\tQ, \tD)$ using split-octonions. 



\begin{rmrk} Of course, there is a third way, by ``brute force'', using computer algebra. We do not find it too illuminating but it does produce quickly  a  list of 14 vector fields on $\Rtt$, generating the infinitesimal $\g_2$-action, as given in Cor.~\ref{cor1}  or Cor~.\ref{cor3}  below. 
\end{rmrk}




In a well-known paper of 1910 (the ``5-variable paper"), Cartan proved
the following.


\begin{theorem}[Cartan,  \cite{C_5var}] 

\sn

 

\begin{enumerate}
\item The symmetry algebra of a $(2,3,5)$-distribution on a connected 5-manifold has dimension at most 14, in which case it is isomorphic to the real split-form of
the simple Lie algebra of type $\g_2$.
\item  All $(2,3,5)$-distributions with 14-dimensional symmetry algebra are
locally diffeomorphic. 
\item If the  symmetry algebra of a $(2,3,5)$-distribution 
has dimension $<14$ then it has  dimension at most 7.
\end{enumerate}
\end{theorem}
%
The last statement is sometimes referred to as Cartan's
``submaximality" result for $(2,3,5)$-distribution. A $(2,3,5)$-distribution with the maximal symmetry algebra $\g_2$ is called {\em flat}. 




 


Using Prop.~\ref{prop:sym} and the fact
that $\SLth$ is $8$-dimensional we immediately conclude from Cartan's submaximality result for $(2,3,5)$-distributions  

\begin{theorem} \label{sym}
The symmetry algebra of the $(2,3,5)$-distribution defined by Eqns.~\eqref{eqnss}   is 14-dimensional,
isomorphic to the Lie algebra $\g_2$, containing the Lie  subalgebra isomorphic $\slth$ generated by the linear $\SLth$-action given by equation (\ref{action}). 
\end{theorem}



\begin{rmrk} In fact, Cartan \cite{C1} and Engel \cite{Eng} {\em defined} in 1893 the Lie algebra
$\g_2$ as the symmetry algebra of a certain 
$(2,3,5)$-distribution on an open set in $\R^5$, using formulas similar to our Eqns.~\eqref{eqnss}.
For example, Engel considers in \cite{Eng} the $(2,3,5)$-distribution obtained by restricting $ d\p-\q\times d\q$   to the linear subspace in $\Rtt$ given by $q_3=p^3$. 
\end{rmrk}







