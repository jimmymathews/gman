
\begin{abstract} The ``dancing metric" is a pseudo-riemannian metric $\met$ of signature (2,2) on the space $\M$ of  non-incident point-line pairs  in the real projective plane $\RPt$. The null-curves  of $(\M,\met)$ are given by the ``dancing condition": {\em the  point  is moving  towards a point  on the line, about which the line  is turning}. We  establish  a  dictionary between classical projective geometry (incidence, cross ratio, projective duality, projective invariants of plane curves\ldots) and pseudo-riemannian 4-dimensional conformal geometry (null-curves and geodesics, parallel transport, self-dual null 2-planes, the Weyl curvature,\ldots). There is also an unexpected bonus: by applying a  twistor construction to $(\M,\met)$, a $\G$-symmetry emerges, hidden deep in classical projective geometry. To uncover this symmetry, one needs to refine the ``dancing condition" by a higher-order condition, expressed in terms of the osculating conic along a plane curve. The outcome is a correspondence between curves in the projective plane and its dual, a projective geometry analog of the more  familiar ``rolling without slipping and twisting" for a pair of   riemannian surfaces. 

\end{abstract}






















